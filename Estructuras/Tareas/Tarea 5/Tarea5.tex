\documentclass[paper=letter, fontsize=12pt]{scrartcl}
\usepackage{amsmath, amssymb, latexsym}
\usepackage{anyfontsize}
\usepackage[utf8]{inputenc}
\usepackage[spanish]{babel}
\begin{document}
\parindent=0mm:
	\begin{center}
  Estructuras Discretas 2017-1\\
  Tarea 5\\
\end{center}

Alumno: Luis Alberto Martinez Monroy\\
N cuenta: 314212391\\ \\

{\bf 1.- Establezca la validez de los sigueintes argumentos usando sólo las principales reglas de inferencia y especificando que regla se usa en cada paso (*)}\\

a)$(p\wedge (p\rightarrow q)\wedge (\neg q \vee r))\rightarrow r$
\begin{enumerate}
	\item$(p\wedge (p\rightarrow q)\wedge (\neg q \vee r)$ PREMISA
	\item$p \wedge (p\rightarrow q)$ $Por E\wedge  (1)$
	\item$(\neg q \vee r)$ $Por E\wedge  (1)$
	\item$p $$Por E\wedge  (2)$
	\item$(p\rightarrow q)$ $Por E\wedge  (2)$
	\item$q$ $Por MP (4)y(5)$
	\item$r$ $Por SD (3)y(6)$
\end{enumerate}

b)$p\wedge q, p\rightarrow (r\wedge q), r\rightarrow (s\vee t), \neg s \therefore t $\\
\begin{enumerate}
	\item$p\wedge q$ PREMISA
	\item$p\rightarrow (r\wedge q)$ PREMISA
	\item$r\rightarrow (s\vee t)$ PREMISA
	\item$\neg s$ PREMISA
	\item$p $ $Por E\wedge (1)$
	\item$(r\wedge q)$ Por MP (1) y (5)
	\item$r$ $Por E\wedge (6)$
	\item$(s\vee t)$ Por MP (3) y (7)
	\item$t$ Por SD (4) y (8)
\end{enumerate}

c)$p\rightarrow (q\rightarrow r), \neg q \rightarrow \neg p, p \therefore r {\bf(**)}$
\begin{enumerate}
	\item$p\rightarrow (p\rightarrow r)$ PREMISA
	\item$\neg q \rightarrow \neg p$ PREMISA
	\item$P$ PREMISA
	\item$(q\rightarrow r)$ Por MP (1) y (3)
	\item$\neg p \vee r$ Por Inferencia especial (**) (2) y (4)
	\item$r$ Por SD (3) y (5)	
\end{enumerate}
(**)= $A\rightarrow B, \neg A \rightarrow C / B\vee C$\\

d)$p\wedge q, r, p\wedge r\rightarrow s \therefore s$\\
\begin{enumerate}
	\item$p\wedge q$ PREMISA
	\item$r$ PREMISA
	\item$p\wedge r\rightarrow s$ PREMISA
	\item$p $ $Por E\wedge (3)$
	\item$r\rightarrow s$ $Por E\wedge (3)$
	\item$s$ Por MP (2) y (5)
\end{enumerate}

e)$p\vee(\neg q \wedge s)\rightarrow s, t, (s\wedge t)\rightarrow u, u\rightarrow \neg w, (\neg p \wedge \neg s)\rightarrow (l \vee \neg k), i \rightarrow (\neg l \wedge k), w \therefore \neg i$\\
\begin{enumerate}
	\item$p\vee(\neg q \wedge s)\rightarrow s$ PREMISA
	\item$t$ PREMISA
	\item$(s\wedge t)\rightarrow u$ PREMISA
	\item$u\rightarrow \neg w$ PREMISA
	\item$(\neg p \wedge \neg s)\rightarrow (l \vee \neg k)$ PREMISA
	\item$i \rightarrow (\neg l \wedge k)$ PREMISA
	\item$w$ PREMISA
	\item$(s\wedge t)\rightarrow \neg w$ Por SH (3) Y (4)
	\item$\neg (s\wedge t)\equiv \neg s \vee \neg t$ Por MT (7) y (8) y equivalente por ley de morgan
	\item$\neg s$ Por SD (9) y (2)
	\item$\neg(p \vee \neg(q \wedge s)) \equiv \neg p \wedge \neg (\neg q \wedge s)$ Por MT (1) y (10) y equivalencia por ley de morgan
	\item$\neg p$ Por E$\wedge$ de (11)
	\item$\neg p \wedge \neg s$ Por I$\wedge$ de (12) y (10)
	\item$(l\vee \neg k)\equiv (\neg \neg l \vee \neg k)\equiv \neg(\neg l \wedge k)$ Por MP (13) y (5)
	\item$\neg i$ Por MT (14) y (6)
\end{enumerate}

{\bf 2.- Demostrar las siguientes propiedades:}\\
a) sea $\Gamma$, $\phi$ un conjunto de formulas y $P$ una expresión lógica, demuestre que si $\Gamma \models P$ y $\Gamma \subseteq \phi$ entonces $\phi \models P$.\\

$\Gamma \models P,\Gamma \subseteq \phi \therefore \phi \models P $\\

Tenemos que la premisa $\Gamma \models P$, y sabemos que hay una propiedad de consecuencia lógica que dice que $A \in \Gamma \rightarrow \Gamma \models A$.\\
Por lo que podemos decir que es necesario $\Gamma \models P$ para $P\in \Gamma$.\\
Si supodemos que $\Gamma \models P$ es cierta, necesariamente $P\in \Gamma$ es cierta.\\

Tenemos que la premisa $\Gamma \subseteq \phi$, dado que $\Gamma y \phi$ es un conjunto de formulas, podemos interpretar que $\Gamma \subseteq \phi$ nos dice que para el conjunto de formulas $\phi$, $\Gamma$ es un elemento, lo que quiere decir que $\Gamma$ es subconjunto de $\phi$.\\

Sabemos que $\phi=\{\{\Gamma \}\}$\\
Sabemos que $P\in \Gamma$\\
Sabemos que $\Gamma=\{P_{0}\wedge...\wedge P\}$\\
Por lo que podemos deducir que$ \phi=\{\{P_{0}\wedge...\wedge P \}\} $\\
y  deducimos $\Gamma \in phi$, por lo tanto $P \in \phi$.\\
Por ultimo concluimos que como $P\in \phi, \phi \models P$\\

b) Sea $\Gamma$, $\phi$, $\Delta$ un conjunto de fórmulas, demuestre que: Si $\Gamma \models \phi$ y $\phi \models \Delta$ entonces $\Gamma \models \Delta$\\

Ya sabemos que para que se cumpla $\Gamma \models A$, es necesario $A \in \Gamma$, por lo que sustituimos y a $A$ por $\phi $ y nos queda que $\phi \in \Gamma$.\\Aplicamos la misma formula $A$ $\in$ $\Gamma$ y cambiamos $A$ por $\Delta$ y $\Gamma$ por $\phi$, lo que nos queda $\Delta \in \phi$. Ya sabiendo esto podemos decir que 
al ser $\Delta$ un elemento de $\phi$ y $\phi$ un elemento de $\Gamma$, podemos deducir que $\Delta \in \Gamma$.\\
Por lo que concluimos que al ser $\Delta$ un elemento de $\Gamma$, $\Gamma \models \Delta$
\end{document}
