\documentclass{book}
\textheight=19cm
\textwidth=14cm
\topmargin=-2cm
\usepackage{amsmath, amssymb, latexsym}

\begin{document}
Alumno: Angela Janin Angeles Martinez\\
Alumno: Luis Alberto Martinez Monroy\\
Estructuras Discretas\\
\begin{center}
  Practica 1\\ \\
  
\end{center}
1. Definicion de conjunto:\\
Una coleccion de elementos que comparten caracteristicas similares.\\ \\

2.Definicion de proposicion:\\
Una proposicion es un enunciado que puede calificarse como verdadero o falso dependiendo del estado en que se evalue.Una proposicion es atomica si no puede subdividirse en proposiciones mas simples y las que no son atomicas se llaman compuestas y los conectores usados para formar proposiciones compuestas se llaman conectivos.\\ \\

3.Definicion de union de dos conjuntos:\\
Es una operacion que resulta en otro conjunto, cuyos elementos son los elementos de los conjuntos iniciales.\\
\end{document}

