\documentclass[paper=letter, fontsize=12pt]{scrartcl}

\usepackage[T1]{fontenc}
\usepackage[utf8]{inputenc}
\usepackage[english, spanish]{babel}
\usepackage{hyperref}
\usepackage{amsfonts,amsthm}
\usepackage{amsmath}
\usepackage{listings}
\usepackage[dvipsnames]{xcolor}
\usepackage{sectsty}


\allsectionsfont{\raggedright\large \textit\normalfont\scshape\emph}

\title{Lineamientos de entrega de prácticas de laboratorio}

\subtitle{
  Estructuras Discretas, 2017-1\\
  Facultad de Ciencias, UNAM
}

\author{
  \normalsize
  Laura Freidberg Gojman\\
  \normalsize
  \texttt{\href{mailto:lfreidberg@yahoo.com}{lfreidberg@yahoo.com}}
  \and
  \normalsize
  José Ricardo Rodríguez Abreu\\
  \normalsize
  \texttt{\href{mailto:ricardo_rodab@ciencias.unam.mx}{ricardo\_rodab@ciencias.unam.mx}}
  \and
  \normalsize
  Albert Manuel Orozco Camacho\\
  \normalsize
  \texttt{\href{mailto:alorozco53@ciencias.unam.mx}{alorozco53@ciencias.unam.mx}}
}

\date{\today}

\begin{document}

\maketitle

\section*{Lineamientos generales}

\begin{itemize}
\item Todas las prácticas se entregarán de manera \emph{individual} a menos que se permita lo contrario.
\item Cualquier sospecha de copia en el código fuente de la práctica será motivo de anulación de la\
  misma para \textbf{TODOS} los que entreguen algo \emph{demasiado} similar.
\item Queda prohibido copiar y pegar texto de tamaño \emph{considerable} de Internet. Si se pide un\
  resumen o paráfrasis de algún tema, es preferible que se haga así con las palabras de cada\
  estudiante a copiar explícitamente algo (especialmente si se detecta el uso de una fuente no\
  fidedigna). En este caso, se procederá a bajar \textbf{2} puntos por cada copia de Internet.
\item El alumno puede hacer citas textuales de tamaño pequeño, siempre y cuando incluya la\
  referencia bibliográfica de dichas citas. La ausencia de lo anterior será penalizado con\
  \textbf{1} punto menos.
\item Para las prácticas de \LaTeX:
  \begin{itemize}
  \item El alumno deberá de usar \emph{su propio} formato de entrega de prácticas que elaborará\
    durante la primera, a menos que se indique lo contrario.
  \item El alumno deberá \textbf{documentar} todas las funcionalidades pedidas en cada práctica.\
    En cada comentario, deberá de explicar brevemente qué hace el bloque de código escrito.\
    (Un buen ejemplo de este punto es observar el código fuente de este documento.)
  \item El código fuente \texttt{*.tex} deberá de de estar dentro de una carpeta llamada\
    \texttt{tex/}.
  \item Queda prohibido el uso de \emph{plantillas} (templates) para mejorar la presentación
    de la práctica.
  \item El alumno deberá de incluir todos los gráficos e imágenes que desee usar dentro de un\
    directorio llamado \texttt{img/}.
  \item Se debe de entregar un comprimido en el cual se incluyan sólo archivos \texttt{.tex},\
    excluyendo \texttt{.pdf} y los autogenerados por el compilador. Además de imágenes, en caso
    de existir.
  \item En general, la estructura de entrega es la siguiente:\\
    \texttt{
      practicaNN/ - |\\
      \hspace*{2.82 cm} | - tex/ - *.tex\\
      \hspace*{2.82 cm} |\\
      \hspace*{2.82 cm} | - img/ - *.(jpg | png | gif | $\ldots$)
    }\\
    donde $NN$ es el número de la práctica.
  \end{itemize}
\item Para las prácticas en Haskell:
  \begin{itemize}
  \item El código deberá de estar bien comentado, inclyendo un comentario por cada función,\
    tipo de dato y constante definida.
  \item El alumno debe de incluir un archivo \texttt{README.tex}, hecho en \LaTeX, con el fin\
    de explicar todas las funcionalidades del programa entregado, así como dificultades al
    programar.
  \item En general, la estructura de entrega es la siguiente:\\
    \texttt{
      practicaNN/ - |\\
      \hspace*{2.82 cm} | - tex/ - README.tex\\
      \hspace*{2.82 cm} |\\
      \hspace*{2.82 cm} | - bin/ - *.hs
    }\\
    donde $NN$ es el número de la práctica.
  \end{itemize}
\item La entrega de las prácticas es exclusivamente vía electrónica, enviando un correo a\
  \texttt{\href{mailto:alorozco53@ciencias.unam.mx}{alorozco53@ciencias.unam.mx}} con\
  asunto \texttt{[Est Disc 2017-1] Práctica NN}.
\item El alumno tiene hasta las $23:59$ (UTC-6, UTC-5 en horario de verano) del día de entrega\
  de la práctica para que le sea aceptada. Posteriormente, el alumno puede entregar de manera\
  tardía con la siguiente penalización:
  \begin{itemize}
  \item $-2$ puntos por $1$ día de retraso
  \item $-3$ puntos por $2$ días de retraso
  \item $-4$ puntos por $3$ días de retraso
  \item \textbf{NO} se aceptará la práctica después de 3 días de retraso.
  \end{itemize}
\end{itemize}

\end{document}
