%% Práctica 1 del curso de Estructuras Discretas, 2017-1
%% Facultad de Ciencias, UNAM
%% Profa: Laura Freidberg Gojman
%% Ayudante: José Ricardo Rodríguez Abreu
%% Ayudante de lab: Albert Manuel Orozco Camacho

%% Author: AlOrozco53
%% Version: 0.0

%% Comienza el preámbulo del documento
\documentclass[paper=letter, fontsize=12pt]{scrartcl}

\usepackage[utf8]{inputenc} %% codificación para la inserción acentos y otros carácteres especiales
\usepackage[T1]{fontenc} %% codificación para la impresión de acentos y otros carácteres especiales
\usepackage[english, spanish]{babel} %% idioma del documento

%% Los siguientes paquetes sirven para utilizar símbolos especiales (principalmente de matemáticas)
\usepackage{amsmath}
\usepackage{textcomp}
\usepackage{marvosym}
\usepackage{wasysym}
\usepackage{amssymb}

\usepackage{hyperref} %% paquete para manejo de hipervínculos
\usepackage{listings} %% paquete para insertar código fuente

\usepackage{sectsty} %% paquete para cambiar el estilo de encabezados de sección

\parindent=5mm %% sangría
\parskip=2mm %% espaciado

%% Formato de fuentes de encabezados de secciones
\allsectionsfont{\raggedright\large \textit\normalfont\scshape\emph}

%% Título
\title{Práctica 1}

%% Subtítulo
\subtitle{
  Estructuras Discretas, 2017-1\\
  Facultad de Ciencias, UNAM
}

%% Autores
\author{
  \normalsize
  Laura Freidberg Gojman\\
  \normalsize
  \texttt{\href{mailto:lfreidberg@yahoo.com}{lfreidberg@yahoo.com}}
  \and
  \normalsize
  José Ricardo Rodríguez Abreu\\
  \normalsize
  \texttt{\href{mailto:ricardo_rodab@ciencias.unam.mx}{ricardo\_rodab@ciencias.unam.mx}}
  \and
  \normalsize
  Albert Manuel Orozco Camacho\\
  \normalsize
  \texttt{\href{mailto:alorozco53@ciencias.unam.mx}{alorozco53@ciencias.unam.mx}}
}

%% Fecha de hoy
\date{\today}
%% Fin del preámbulo del documento

%% Comienza el cuerpo del documento
\begin{document}

%% Creación del título
\maketitle

\section*{Objetivo de la práctica}

\noindent
El alumno aprenderá a identificar las herramientas básicas que \LaTeX{} provee para dar\
formato a un documento, comenzando con la primera página. El alumno, además, aprenderá\
a manejar dichas herramients al utilizarlas para crear una plantilla (formato) de entrega\
de tareas.

\section*{Preámbulo}

\noindent
Para usar cualquier procesador de texto, es de vital importancia conocer las formas en que\
podemos modificar un documento a nuestra conveniencia. En el caso de \LaTeX, estas formas\
se reducen a importar paquetes y utilizar comandos, en la mayoría de los casos.\par
Debido a que \LaTeX{} contiene una enorme cantidad de clases, paquetes y comandos, es imposible\
presentarlos todos en un libro (como el que está en la página del curso) o en notas. Por\
ello, es importante familiarizarse con los diferentes sitios web que proveen ayuda instantánea\
tan solo con hacer una búsqueda en Google.

\section*{Actividades a realizar}

\noindent
Utilizando lo visto en clase (presentación y libro de consulta), deberán diseñar y elaborar un\
formato de entrega de tareas (plantilla) que deberá incluir lo siguiente:
\begin{itemize}
\item datos de la universidad y de la facultad
\item datos de la asignatura
\item título del documento
\item subtítulo (opcional)
\item nombre de los profesores (opcional)
\item nombre del alumno y número de cuenta
\item fecha de entrega del documento
\item formato de párrafos
\item formato de título de secciones
\item índice (opcional)
\item numeración de página
\item pie de página (opcional)
\end{itemize}\par
\emph{
  Observación: con \textbf{tarea} no se hace referencia a las demás tareas que se entregarán\
  en el curso (teóricas), sino a las prácticas en \LaTeX{} y los archivos README de las prácticas\
  en Haskell.
}\par
La plantilla deberá de contener todos los datos principales del documento en una \textbf{portada}
que estará al principio y aparte del resto del documento.\par
Para ilustrar el formato que le darán al cuerpo del documento, deberán realizar lo siguiente:
\begin{enumerate}
\item Escribir qué es un fractal.
\item Escribir cuatro situaciones del ``mundo real'' en las que aparece\
  la sucesión de Fibonacci.
\item Escribir qué tienen en común los fractales y la sucesión de fibonacci.
\item Escribir una lista de las distribuciones de Linux más importantes, mostrando una\
  jerarquía de distros \textit{distros}. Como referencia, pueden consultar\
  \texttt{\href{https://en.wikipedia.org/wiki/List_of_Linux_distributions}{este enlace}}\
  (¡mas no pueden citar directamente de ahí ya que es Wikipedia!).
\item Escribir un comentario y una reseña sobre alguno de los videos en los siguientes enlaces:
  \begin{itemize}
  \item \texttt{\href{https://www.youtube.com/watch?v=dNRDvLACg5Q}{Explicación informal sobre máquinas de Turing}}\
    (disponible sólo en inglés y sin subtítulos)
  \item \texttt{\href{https://www.youtube.com/watch?v=ONs_qejnD8c}{¿Se nos acabarán algún día los nombres?}}\
    (disponible en inglés y con subtítulos en español)
  \end{itemize}
\end{enumerate}\par
Deberán de escribir todas las fuentes bibliográficas consultadas en una bibliografía ubicada al\
final de su documento \texttt{.tex} utilizando los comandos\\
\verb+\begin{thebibliography}  ... \end{thebibliography}+.\par
Por ejemplo, pueden hacer algo como lo siguiente:
\begin{verbatim}
\begin{thebibliography}{99}
\bibitem{Goossens} M. Goossens; F, Mittelbach; A. Samarin.
                     {\it The \LaTeX Companion}. Addison-Wesley. 1993.
\bibitem{Lamport} L. Lamport. {\it \LaTeX}. Addison-Wesley. 1996.
\end{thebibliography}
\end{verbatim}
para producir:
\begin{thebibliography}{99}
\bibitem{Goossens} M. Goossens; F, Mittelbach; A. Samarin.
  {\it The \LaTeX Companion}. Addison-Wesley. 1993.
\bibitem{Lamport} L. Lamport. {\it \LaTeX}. Addison-Wesley. 1996.
\end{thebibliography}\par
Nótese que el parámetro \texttt{99} indica el número \emph{máximo} de referencias a insertar y\
que cada referencia va precedida por un comando \verb+\bibitem{...}+.

\section*{Sugerencias}

\noindent
Para la realización de la plantilla, pueden consultar ejemplos como los siguientes:
\begin{itemize}
\item \texttt{\href{https://www.overleaf.com/gallery/tagged/homework\#.V7pPbGXMzFI}
  {plantillas en overleaf.com}}
\item \texttt{\href{https://www.dropbox.com/s/cxenfopmt9qzfk1/assignment_3.tex?dl=0}
  {una plantilla que uso para mis tareas}}
\end{itemize}\par
Revisen el libro que está en la página del curso y las diapositivas que les (intenté) presentar\
el pasado viernes. Recuerden que pueden buscar y anotar quién es el personaje que viene en la\
primera diapositiva de dicha presentación y qué ha contribuido al mundo de la computación\
y/o programación para ganar un punto extra.

\section*{Entrega}

La entrega es por correo electrónico siguiendo los lineamientos establecidos en el documento que\
se encuentra en la página del curso. Incluyan en el mismo cualquier comentario, inquietud\
o dificultad al realizar la práctica.\par
La fecha de entrega es el próximo \emph{domingo 21 de agosto de 2016} antes de las \emph{23:59 hrs}.

\end{document}
%% Fin del cuerpo del documento
