%% Práctica 2 del curso de Estructuras Discretas, 2017-1
%% Facultad de Ciencias, UNAM
%% Profa: Laura Freidberg Gojman
%% Ayudante: José Ricardo Rodríguez Abreu
%% Ayudante de lab: Albert Manuel Orozco Camacho

%% Author: AlOrozco53
%% Version: 0.5

%% Comienza el preámbulo del documento
\documentclass[paper=letter, fontsize=12pt]{scrartcl}

\usepackage[utf8]{inputenc} %% codificación para la inserción acentos y otros carácteres especiales
\usepackage[T1]{fontenc} %% codificación para la impresión de acentos y otros carácteres especiales
\usepackage[english, spanish]{babel} %% idioma del documento

%% Los siguientes paquetes sirven para utilizar símbolos especiales (principalmente de matemáticas)
\usepackage{amsmath}
\usepackage{textcomp}
\usepackage{marvosym}
\usepackage{wasysym}
\usepackage{amssymb}

\usepackage{hyperref} %% paquete para manejo de hipervínculos
\usepackage{listings} %% paquete para insertar código fuente

\usepackage{sectsty} %% paquete para cambiar el estilo de encabezados de sección

\parindent=5mm %% sangría
\parskip=2mm %% espaciado

%% Formato de fuentes de encabezados de secciones
\allsectionsfont{\raggedright\large \textit\normalfont\scshape\emph}

%% Título
\title{Práctica 2}

%% Subtítulo
\subtitle{
  Estructuras Discretas, 2017-1\\
  Facultad de Ciencias, UNAM
}

%% Autores
\author{
  \normalsize
  Laura Freidberg Gojman\\
  \normalsize
  \texttt{\href{mailto:lfreidberg@yahoo.com}{lfreidberg@yahoo.com}}
  \and
  \normalsize
  José Ricardo Rodríguez Abreu\\
  \normalsize
  \texttt{\href{mailto:ricardo_rodab@ciencias.unam.mx}{ricardo\_rodab@ciencias.unam.mx}}
  \and
  \normalsize
  Albert Manuel Orozco Camacho\\
  \normalsize
  \texttt{\href{mailto:alorozco53@ciencias.unam.mx}{alorozco53@ciencias.unam.mx}}
}

%% Fecha de hoy
\date{\today}
%% Fin del preámbulo del documento

%% Comienza el cuerpo del documento
\begin{document}

%% Creación del título
\maketitle

\section*{Objetivo de la práctica}

\noindent
El alumno ejercitará el uso de las diversas herramientas para imprimir símbolos, operaciones,\
ecuaciones y figuras matemáticas. Además, tendrá la posibilidad de establecer definiciones,\
axiomas, teoremas y corolarios, así como demostrar éstos últimos.

\section*{Preámbulo}

\noindent
Probablemente la característica que hace que \LaTeX{} sea tan popular en la comunidad académica\
es la facilidad que brinda al usuario de incorporar sintaxis técnica a documentos. En particular,\
los que radicamos dentro de o para alguna carrera del Departamento de Matemáticas de la Facultad\
de Ciencias, nos interesamos en simbología matemática.

\section*{Actividades a realizar}

\noindent
Utilizando lo visto en clase (presentación y libro de consulta), realizar lo siguiente,\
especificando paso por paso:
\newpage
\begin{enumerate}
\item $[\frac{1}{10}\mbox{ pts.}]$ Formalice el siguiente argumento usando variables proposicionales:\par
 \emph{``Si un unicornio es mítico también es inmortal, pero si no lo es, entonces es un mamífero mortal.\
  Si un unicornio es inmortal o mamífero tiene cuerno. Un unicornio es mágico si tiene un cuerno.\
  Luego entonces simpre un unicornio es mágico y tiene cuerno.''}
\item $[\frac{1}{10}\mbox{ pts.}]$ Probar que el argumento anterior es válido (satisfacible)\
  utilizando interpretaciones.
\item $[\frac{1}{10}\mbox{ pts.}]$ Decida si la siguiente equivalencia lógica es válida o no,\
  utilizando las leyes vistas en clase y justificando paso por paso:
  \[
  (p \wedge  (q \to r)) \to r \wedge (p \vee s)
  \equiv
  \neg p \vee (q \wedge \neg r) \vee (r \wedge p) \vee (r \wedge s)
  \]
\item $[\frac{1}{8}\mbox{ pts.}]$ Resuelva las siguientes ecuaciones y desigualdades, indicando paso por paso las\
  acciones que va realizando. Se sugiere fuertemente que se consulten las páginas $57$ a $59$ del\
  libro. Escoja una ecuación de las siguientes y explique el procedimiento dentro de un párrafo.\
  (Si alguna de las siguientes, no tiene solución real, especifique las soluciones complejas.)
  \begin{itemize}
  \item $36x^2 - 80x + 23 = 17x - 6$.
  \item $\displaystyle{\sqrt{4x^2 - 28x + 49} \leq \sqrt[3]{x^{3 + \log_x 27}} + 38}$.
  \item $\displaystyle{\frac{1 + \sin x}{\cos x} + \frac{\cos x}{1 + \sin x} = 4}$. Especifique la(s) solucion(es)\
    exacta(s).
  \item
    $\left\{
    {{
      3x + 2x - 10 = 8y
      \atop
      9x + 3 = -24y + 5 - z
   } }\atop
      10z - 2x + 3 = 0
    \right.$
  \end{itemize}
\item $[\frac{1}{10}\mbox{ pts.}]$ Escriba el sistema de ecuaciones del inciso anterior en forma matricial $AX = B$ (Busque en su libro\
  de Álgebra Superior I o en uno de Álgebra Lineal o consúltelo con su profesor de la materia y cite\
  su fuente.)
\item $[\frac{1}{20}\mbox{ pts.}]$ Defina \emph{formalmente} el concepto de \emph{límite de una función real} usando $\epsilon$'s y $\delta$'s.
\item $[\frac{1}{10}\mbox{ pts.}]$ Calcule lo siguiente, justificando paso por paso:
  \begin{itemize}
  \item $\lim_{x \to -2}\frac{3x^2 - 8x - 3}{2x^2 - 18}$
  \item $\frac{d}{dx} \left[ \frac{5x^2 - \ln (\cos x)}{8x^3 - 10x} \right]$
  \item $\int_0^1 (x^6 - 2x^3 + 5x - 3) dx$
  \end{itemize}
\item $[\frac{1}{10}\mbox{ pts.}]$ Enuncie los axiomas de Peano.
\item $[\frac{1}{10}\mbox{ pts.}]$ Enuncie la paradoja de Russell.
\item $[\frac{1}{4}\mbox{ pts.}]$ Enuncie y demuestre la siguiente proposición (para cualquier conjunto $A$):
  \[ A \bigcap A^C = \emptyset.\]
\item $[\frac{1}{8}\mbox{ pts.}]$ Enuncie una de las propiedades distributivas para cualesquiera tres conuntos y\
  demuéstrela. (\textbf{NOTA}: Su prueba deberá de incluir un argumento en el que use\
  dos contenciones de conjuntos.)
\end{enumerate}

\section*{Sugerencias}

Pueden ayudarse entre varias personas para realizar los ejercicios anteriores. Sin embargo,\
la entrega es individual.\par
Recuerden que pueden buscar y anotar quién es el personaje que viene en la\
primera diapositiva de dicha presentación y qué ha contribuido al mundo de la computación\
y/o programación para ganar un punto extra.

\section*{Uso de bibliografía}
Deberán de escribir todas las fuentes bibliográficas consultadas en una bibliografía ubicada al\
final de su documento \texttt{.tex} utilizando los comandos\\
\verb+\begin{thebibliography}  ... \end{thebibliography}+.\par
Por ejemplo, pueden hacer algo como lo siguiente:
\begin{verbatim}
\begin{thebibliography}{99}
\bibitem{Goossens} M. Goossens; F, Mittelbach; A. Samarin.
                     {\it The \LaTeX Companion}. Addison-Wesley. 1993.
\bibitem{Lamport} L. Lamport. {\it \LaTeX}. Addison-Wesley. 1996.
\end{thebibliography}
\end{verbatim}
para producir:
\begin{thebibliography}{99}
\bibitem{Goossens} M. Goossens; F, Mittelbach; A. Samarin.
  {\it The \LaTeX Companion}. Addison-Wesley. 1993.
\bibitem{Lamport} L. Lamport. {\it \LaTeX}. Addison-Wesley. 1996.
\end{thebibliography}\par
Nótese que el parámetro \texttt{99} indica el número \emph{máximo} de referencias a insertar y\
que cada referencia va precedida por un comando \verb+\bibitem{...}+.

\section*{Entrega}

La entrega es por correo electrónico siguiendo los lineamientos establecidos en el documento que\
se encuentra en la página del curso. Incluyan en el mismo cualquier comentario, inquietud\
o dificultad al realizar la práctica.\par
La fecha de entrega es el próximo \emph{lunes 12 de septiembre de 2016} antes de las \emph{23:59 hrs}.

\end{document}
%% Fin del cuerpo del documento
